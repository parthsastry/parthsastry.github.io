

\projectentry{
    Fluid Flows and Lagrangian Coherent Structures
} % title
{
    Advisor: Prof. Punit Parmananda, Department of Physics,
    IIT Bombay \\
    \emph{Course Project}
}
{
    Jul'19-nov'19
}
{
    \begin{itemize}
    \item Looked at and made some models and simulations   for fluid flows, and understood the idea of \textbf{Lagrangian Coherent Structures}
    \item Presented as part of our Honors class, on the topic, and studied about the \textbf{various types and ways to model LCSs}
    \item Looked at \textbf{examples of LCSs and their structure}, as seen on various planets in our Solar System
    \item Studied about ways to obtain data for modeling, and \textbf{used sample datasets to make model simulations of fluid flow and evolution of LCSs}
    \end{itemize}
}

\projectentry{
    Black Hole Information Paradox
} % title
{
    Advisor: Prof. S. Mohanty, Physical Research Laboratory, Ahmedabad \\
    \emph{Supervised Reading}
}
{
    May'19-Jul'19
}
{
    \begin{itemize}
        \item Studied about the \textbf{Black Hole Information Paradox} and about the modern approach to the Paradox proposed by Stephen Hawking
        \item Learned about associated concepts like \textbf{Quantum Field Theory, General Relativity, and Penrose Diagrams}
        \item Learned about modern resolutions to the paradox, like the \textbf{Black Hole Firewall, and the Fuzzball}
        \item Built a background to understand and look at the problem and associated problems through the lens of String Theory and Quantum Electrodynamics
    \end{itemize}
}

\projectentry{
    Digital Stopwatch
} % title
{
    Advisor: Prof. Mahesh B. Patil, Department of Electrical Engineering, IIT Bombay \\
    \emph{Course Project}
}
{
    Feb'2019-Apr'19
}
{
    \begin{itemize}
        \item Alongside two others, built a \textbf{functioning stopwatch with a start/stop/reset functionality} capable of measuring time with a precision of one second up to one hour
        \item Utilized \textbf{IC555 timers, and IC7041 decade counters}, with BCD to 7-Segment displays to design our circuit
        \item Enabled the resetting of the minutes display, allowing us to extend our circuit design to have hour display
    \end{itemize}
}
